% This is a -*- latex -*- file

\documentclass[11pt]{article}

\title{PolyJ 1.2 Translation}
\author{Jed Liu \and Grant Wang\footnote{Translated to \LaTeX\ by Aleksey Kliger}}
\date{11 August 2000\footnote{Submitted to Andrew Myers}}

\begin{document}
\maketitle

\begin{abstract}
A summary of the design of the translation phase of the PolyJ 1.2
compiler.  Highlights differences between the PolyJ 1.0.2 and PolyJ 1.2
compilers.  Summarizes the current status of the PolyJ 1.2 implementation.
\end{abstract}

\tableofcontents

\section{Adapter Interfaces and Classes}

Translation in PolyJ 1.2 occurs exactly as in PolyJ 1.0.2, except that
the abstract adapter classes (e.g., \verb+Vector$$$W+) are now adapter
interfaces.  This was done to avoid having an adapter class pointer at
every level in the class hierarchy.  This change has a few
repercussions.

\begin{itemize}
\item The \verb+$cast+ and \verb+$instanceof+ methods have been moved
into the Java translations of the .pj files.  These methods are also
no longer static, so as to support nested classes.
\item The class hierarchy of the polymorphic classes is mirrored in the
adapter classes. (e.g., if \verb+A+ is a subclass of \verb+B+ then
\verb+A$$$W+ is a sub-interface of \verb+B$$$W+)
\item Only the polymorphic classes at the top of the class hierarchy
have the \verb+$W+ adapter pointer.  This means that whenever we need
to use an adapter object, we will need to cast it to the
appropriate level in the adapter class hierarchy.
\item The \verb-$PSig-, \verb-$FetchS- and \verb-$SetS- methods are now
flagged with class anmed.  (e.g., if we have \verb-A[int]-, then the
adapter interface \verb-A$$$W- will have an abstract method named
\verb-A$$$W$PSig$1- and in the \verb-A$$int- implementation of this
interface, the method would return the string ``\verb-I-''.)
\end{itemize}

\section{Translation Infrastructure}

Translation in PolyJ $1.2$ is handled by a translation visitor that
extends the existing translation visitor in Polyglot.  The PolyJ
translation visitor first walks over the AST and produces a Java
translation of the .pj file.  As it does this, the visitor saves
information to three data structures so that it can generate the
appropriate adapter interfaces and classes once the .pj file is
translated.  The classes for these data structures are kept in the
package \verb-polyglot.ext.polyj.util-.

The first of these data structures, \verb-WhereBindingTable-, holds
information about the method constraints found in the where clauses of
the .pj file.  The member field of interest in this class is the
Hashtable \verb-whereBindings-.  This object is a Hashtable of
Hashtables of Hashtables of \verb-WhereBindingTableEntry- objects,
making for a three-dimensional table of \verb-WhereBindingTableEntry-
objects.  Each of these \verb-WhereBindingTableEntry- objects holds data
about a single method constraint and the where method to which it is
bound.  A \verb-WhereBindingTableEntry- for any given method constraint can
be obtained from the table by indexing into it three times: first, by
the fully-qualified name of the class constrained by the method;
second, by the name of the parameter to which the constraint applies;
and finally, by an encoding of the method signature of the constraint.

Method signatures are encoded into a dash-delimitated string, where
first we have the fully-qualified name of the return type, followed by
the method name, followed by the fully-qualified names of the formal
arguments in order.  For example, if \verb-X- is a class parameter,
then the method constraint
``\verb-void insertElementAt(X x, int pos)-'' would be encoded as
``\verb+void--insertElementAt--X--int+''.  For this encoding,
constructors are considered to have void return types.

The second data structure, \verb-InstantiationTable-, holds
information about instantiations of polymorphic types found in the .pj
file.  The member field of interest in this class is the Hashtable
\verb-instantiationTable-.  This object is a Hashtable of
\verb-InstantiationTableEntry- objects, indexed by the fully-qualified names
of polymorphic classes.  Each \verb-InstantiationTableEntry- holds
information on the instantiations of that particular polymorphic
class, having a Hashtable that maps the fully-qualified name of an
adapter class to its corresponding PJInstantiatedType object.

The last data structure is \verb-StaticFieldBindingTable-.  As its
name suggests, this data structure holds information about the static
field members of the polymorphic types found in the .pj file.  The
member field of interest in this class is the Hashtable
\verb-fieldBindings-.  This object is a Hashtable of Hashtables of
\verb-StaticFieldBindingTableEntry- objects, making for a
two-dimensional table of \verb-StaticFieldBindingTableEntry- objects.
Each of these \verb-StaticFieldBindingTableEntry- objects holds data
about a single static field and the Fetch and Set methods to which it
is bound.  A \verb-StaticFieldBindingTableEntry- for any given static
field can be obtained from the table by indexing into it twice: first,
by the fully-qualified name of the class of which the field is a
member; and second, by the name of the static field itself.

\section{The \emph{Status Quo}}

This is what has been done so far.  The focus has mainly been on
getting the translation of basic programs to work so that we can
expand on language features from there.  As such, we haven't been
focusing on nested classes; it was decided that patching translation
to support inner classes should not be difficult.

\subsection{Translation of polymorphic classes}

\paragraph{Changes:}
\verb-$cast-, \verb-$instanceof- generated for polymorphic classes.

\paragraph{Files:}
\begin{itemize}
\item \verb-polyglot.ext.polyj.ast.PolyJClassNode-
\end{itemize}

\subsection{Generation of adapter interfaces and adapter classes}

\paragraph{Changes:}
Adapter interfaces and adapter classes generated.

\paragraph{Files:}
\begin{itemize}
\item \verb-polyglot.ext.polyj.visit.PolyJTranslationVisitor-
\end{itemize}

\subsection{Generation of \texttt{\$FetchS} and \texttt{\$SetS}}

\paragraph{Changes:}
Abstract methods \verb+$FetchS+ and \verb+$SetS+ generated in
adapter interfaces, implemented in adapter classes.

\paragraph{Files:}
\begin{itemize}
\item \verb+polyglot.ext.polyj.visit.PolyJTranslationVisitor+
\end{itemize}

\subsection{Translation of cast and instanceof expressions.}

\paragraph{Changes:}
Translates cast and instanceof expressions to method calls if type
is polymorphic, does not handle types instantiated on
\verb+PJParameterType+

\paragraph{Files:}
\begin{itemize}
\item \verb+polyglot.ext.polyj.ast.PolyJCastExpression+
\item \verb+polyglot.ext.polyj.ast.PolyJInstanceOfExpression+
\end{itemize}

\subsection{Constructor Translation.}
\paragraph{Changes:}
Adds a parameter to constructors if the class is polymorphic, but
does not add the ``\verb+this.$W = $W+'' statement to the body of
the constructor.
\paragraph{Files:}
\begin{itemize}
\item \verb+polyglot.ext.polyj.ast.PolyJMethodExpression+
\end{itemize}

\subsection{Translation infrastructure}
As described previously in this document.

\subsection{Translation of constructor calls (\(\texttt{super}(\ldots)\),
\(\texttt{this}(\ldots)\))}
\paragraph{Changes:}
Adds \verb+$W+ to argument list of call to \verb+this()+ or \verb+super()+
is a call to a polymorphic class constructor.

\paragraph{Files:}
\begin{itemize}
\item \verb+polyglot.ext.polyj.ast.PolyJConstructorCallStatement+
\end{itemize}

\subsection{Field Expression translation.}
\paragraph{Changes:}
If the field is a static expression, it's translated to a fetch call.
Additionally, if the field is known, a cast is done.

\paragraph{Files:}
\begin{itemize}
\item \verb+polyglot.ext.polyj.ast.PolyJFieldExpression+
\end{itemize}

\subsection{Method expressions (including translation of where methods
to \texttt{\$W.\$Where\$x})}
\paragraph{Changes:}
If the method call is a where method, we translate to call the correct
\verb+$W.$Where$x+ method in the adapter class.

\paragraph{Files:}
\begin{itemize}
\item \verb+polyglot.ext.polyj.ast.PolyJMethodExpression+
\end{itemize}

\subsection{Addition of \texttt{PolyJAmbiguousName} and \texttt{PolyJAmbiguousNameExpression}}
So that \verb+PolyJFieldExpression+s are generated during
disambiguating.

\subsection{Modifications to \texttt{polyj.cup}}
To allow generation of PolyJ specific AST nodes.

\section{Known Deficiencies}

\subsection{Translation of local classes is broken.}

When we translate the new object expression, we refer to the fully
qualified class name, when we should just refer to the short name.
Additionally, the adapter classes are not generated with the correct
names.  Translation of new object expressions for member classes is
correct, but, as with local classes, the adapter class name needs to
be mangled, i.e.

\begin{verbatim}
class A[X] {
    class B[X] {}
    void foo() {
        B[Object] bar = new B[Object]();
    }
}
\end{verbatim}

is translated to 

\begin{verbatim}
class A {
    class B { // typechecking methods in here }
    void foo() {
        A.B bar =
            new A.B(pjIns.A.B$$java$lang$Object.getInstance());
    }
}
\end{verbatim}

Note here that the reference to the adapter class is incorrect since the
name should be mangled: instead of \verb+pjIns.A.B$$java$lang$Object+,
it shoudl eb \verb+pjIns.A$B$$java$lang$Object+.


\subsection{Instantiation on \texttt{PJParameterType} is not complete.\label{sec:param}}
We do not reference the \verb+getWhere+ and \verb+PSig+ methods, i.e.
code such as the following does not translate correctly:

\begin{verbatim}
class A[X,Y] {
    void foo() {
        A[X,X] bar = new A[X,X]();
    }
}
\end{verbatim}

should generate something like the following below, but does not.

\begin{verbatim}
class A {
    void foo() {
        A var = new A($W.getWhere$1());
    }
}
\end{verbatim}

Here, we need a binding from types to \verb+getWhere+ methods
in the translation infrastructure.

\subsection{Translation of static fields.}

The \verb+$FetchS+ and \verb+$SetS+ methods are generated in the
adapter classes.  Calls to \verb+$FetchS+ are being made, but
assignments to static fields are broken.  For these, \verb+$SetS+
calls need to be generated.

\subsection{Type descriptors for types such as \texttt{A[String]}
are not produced.}

That is, they are not of the form used by PolyJ 1.0.2.  For example
``\verb+A[String]+'' should translate to ``\verb+LA<java/lang/String;>+'',
but this does not occur.  Therefore, casts such as:
\verb+(A[B[String]])x+ are not translated correctly.  This
corresponds to \ref{sec:param} above, since type descriptors
for \verb+PJParameterType+s come in the form of calls to \verb+$PSig+
methods, such as the following below:

\begin{verbatim}
class A[X,Y] {
    void foo() {
        A[X,X] bar = new A[X,X]();
        ((A[X,X])bar).toString();
    }
}
\end{verbatim}

should be translated to something like:

\begin{verbatim}
class A{
    void foo() {
        A bar = new A($W.GetWhere$1());
        A.$cast(bar, $W.psig$1);
    }
}
\end{verbatim}

\subsection{Translation of constructors in polymorphic classes.}
While signatures are translated correctly, i.e. \verb+public A()+
\(\mapsto\) \verb+public A(A$$$W $W)+, the addition of the statement
``\verb+this.$W = $W+'' is not added to the body of the constructor.


\section{Known Bugs}

This is a list of known bugs elsewhere in the compiler that we have come
across while working on translation.

\subsection{Parsing problem}

The following code does not parse:

\begin{verbatim}
class A[X] {
    void foo() {
        A[B[String]] x = new A[B[String]]();
        ((A[B[String]])x).toString();
    }
}

class B[X] { }
\end{verbatim}

\subsection{Multiple public classes per file.}

The following code compiles, but should not.  We cannot declare two
public classes in a file.  The example code was compiled in a file named
\verb+G.pj+, and yet the code compiled without throwing errors.
This is different from the Sun \verb+javac+ behaviour.

\begin{verbatim}
public class A[X] { }

public class B {
    A[String] x = new A[String]();
}
\end{verbatim}

\subsection{Overloading based on return type.}

\begin{verbatim}
class A[X] 
  where X { boolean foo(Object o);  int foo(Object o); }
{
    X me;

    void foo() {
        int x = 3 + me.foo(�hi�);
    }
}
\end{verbatim}

The above code throws the following error.

\begin{verbatim}
D.pj:5: Additive operators must have numeric operands.
\end{verbatim}

However, if the order of the where methods is switched, the code
compiles fine.  We believe that the compiler should throw a type
checking error, since it is impossible for any class X to have methods
boolean foo(Object o) and int foo(Object o) that differ only in return
type.

\subsection{Accessing static fields as \texttt{instance.field}.}

\begin{verbatim}
interface I {
    public int a = 0;
    public int b = 0;
}

class C implements I { }

class E {
    C c = new C();
    void main() {
        if ( c.a == c.b ) { }
    }
}
\end{verbatim}

The above code generates the following error.

\begin{verbatim}
static.pj:11: Field "a" not found in context C
static.pj:11: Field "b" not found in context C
static.pj: 2 errors.
\end{verbatim}

\end{document}